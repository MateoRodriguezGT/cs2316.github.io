\documentclass[addpoints,9pt]{exam}
\usepackage{verbatim, multicol, tabularx,}
\usepackage{amsmath,amsthm, amssymb, latexsym, listings, qtree}

\lstset{frame=tb,
  language=Java,
  aboveskip=1mm,
  belowskip=0mm,
  showstringspaces=false,
  columns=flexible,
  basicstyle={\ttfamily},
  numbers=none,
  frame=single,
  breaklines=true,
  breakatwhitespace=true
}


\title{CS 2316 Exam 1 Practice}
\date{}
\setcounter{page}{0}
\begin{document}

\maketitle
\thispagestyle{head}
%% \firstpageheader{}
%%               {\tiny Copyright \textcopyright\ 2014 All rights reserved. Duplication and/or usage for purposes of any kind without permission is strictly forbidden.}
%%               {}


\runningheader{}
              {\tiny Copyright \textcopyright\ 2014 All rights reserved. Duplication and/or usage for purposes of any kind without permission is strictly forbidden.}
              {}

\footer{Page \thepage\ of \numpages}
              {}
              {Points available: \pointsonpage{\thepage} -
               points lost: \makebox[.5in]{\hrulefill} =
               points earned:  \makebox[.5in]{\hrulefill}.
              Graded by: \makebox[.5in]{\hrulefill}}


\ifprintanswers
\begin{center}
{\LARGE ANSWER KEY}
\end{center}
\else
\vspace{0.1in}
\hbox to \textwidth{Name (print clearly): \enspace\hrulefill}
\vspace{0.3in}
\hbox to \textwidth{Signature: \hrulefill}

\vspace{0.3in}
\hbox to \textwidth{GT account username (gtg, gth, msmith3, etc): \enspace\hrulefill}

\fi

\vfill

\begin{itemize}
\item Signing signifies you are aware of and in accordance with the {\bf Academic Honor Code of Georgia Tech}.
\item Calculators and cell phones are NOT allowed.
\item This is a Python programming test.  Where asked for Python statements or expressions you must print them exactly as they would be typed in a Python source file or interactive shell.
\end{itemize}

\vfill

% Points Table
%\begin{center}
\addpoints
%\gradetable[v][pages]
%\end{center}

% Points Table
\begin{center}
\renewcommand{\arraystretch}{2}
\begin{tabularx}{\textwidth}{|l|c|X|X|X|}
        \hline
Question & Points per Page & Points Lost & Points Earned & Graded By \\
\hline
Page 1 & \pointsonpage{1} & - & =  &\\
\hline
Page 2 & \pointsonpage{2} & - & =  &\\
\hline
Page 3 & \pointsonpage{3} & - & =  &\\
\hline
Page 4 & \pointsonpage{4} & - & =  &\\
\hline
Page 5 & \pointsonpage{5} & - & =  &\\
\hline
Page 6 & \pointsonpage{6} & - & =  &\\
\hline
Page 7 & \pointsonpage{7} & - & =  &\\
\hline
Page 8 & \pointsonpage{8} & - & =  &\\
\hline
TOTAL & \numpoints & - & =  & \\
\hline
\end{tabularx}
\end{center}

\newpage

%\normalsize

\pointsinmargin
\bracketedpoints

\marginpointname{}
%%%%%%%%%%%%%%%%%%%%%%%%%%%%%%%%%%%%%%%%%%%%%%%%%%%%%%%%%%%%%%%%%%%%%%%%%%%%

\begin{questions}

\question {\bf True or False}

In each of the blanks below, write ``T'' if the statement beside the blank is true, ``F'' otherwise.\\

\begin{parts}

\ifprintanswers
\part[1] \underline{ {\bf T} } Every Python value has a type such as {\tt float} or {\tt int}.\\
\else
\part[1] \makebox[.25in]{\hrulefill} Every Python value has a type such as {\tt float} or {\tt int}.\\
\fi

\ifprintanswers
\part[1] \underline{ {\bf F} } Python variables are statically typed, meaning that once you assign a value to a variable you can only assign new values of the same type.  For example, after {\tt x = 3.14} you can only assign {\tt float} values to {\tt x}.\\
\else
\part[1] \makebox[.25in]{\hrulefill} Python variables are statically typed, meaning that once you assign a value to a variable you can only assign new values of the same type.  For example, after {\tt x = 3.14} you can only assign {\tt float} values to {\tt x}.\\
\fi

\ifprintanswers
\part[1] \underline{ {\bf F} } The {\tt +} operator means the same for {\tt str} values as it does for {\tt int} values.\\
\else
\part[1] \makebox[.25in]{\hrulefill} The {\tt +} operator means the same for {\tt str} values as it does for {\tt int} values.\\
\fi

\ifprintanswers
\part[1] \underline{ {\bf F} } {\tt try = try + 1 \# increment the number of tries} is a valid Python statement.\\
\else
\part[1] \makebox[.25in]{\hrulefill} {\tt try = try + 1 \# increment the number of tries} is a valid Python statement.\\
\fi

% \ifprintanswers
% \part[1] \underline{ {\bf F} } \\
% \else
% \part[1] \makebox[.25in]{\hrulefill} \\
% \fi

% \ifprintanswers
% \part[1] \underline{ {\bf F} } \\
% \else
% \part[1] \makebox[.25in]{\hrulefill} \\
% \fi

% \ifprintanswers
% \part[1] \underline{ {\bf F} } \\
% \else
% \part[1] \makebox[.25in]{\hrulefill} \\
% \fi

% \ifprintanswers
% \part[1] \underline{ {\bf F} } \\
% \else
% \part[1] \makebox[.25in]{\hrulefill} \\
% \fi

% \ifprintanswers
% \part[1] \underline{ {\bf F} } \\
% \else
% \part[1] \makebox[.25in]{\hrulefill} \\
% \fi

% \ifprintanswers
% \part[1] \underline{ {\bf F} } \\
% \else
% \part[1] \makebox[.25in]{\hrulefill} \\
% \fi

\end{parts}

\newpage

\question {\bf Expression Evaluation}

\begin{parts}

For each expression below, write the value and then the Python data type of the evaluated legal expression in the space provided.  Be exact.\\\\

\ifprintanswers
\hspace{-.5in} Expression: {\tt 7 / 2}
\part[1] Calculated value: \underline{ 3.5 }\\
\part[1] Type: \underline{ {\tt float}  }\\\\
\else
\hspace{-.5in} Expression: {\tt 7 / 2}
\part[1] Calculated value: \makebox[1in]{\hrulefill}\\
\part[1] Type: \makebox[1in]{\hrulefill}\\\\
\fi

\ifprintanswers
\hspace{-.5in} Expression: {\tt 64 - 16 * 2}
\part[1] Calculated value: \underline{ 32  }\\
\part[1] Type: \underline{ {\tt int}  }\\\\
\else
\hspace{-.5in} Expression: {\tt 64 - 16 * 2}
\part[1] Calculated value: \makebox[1in]{\hrulefill}\\
\part[1] Type: \makebox[1in]{\hrulefill}\\\\
\fi

\ifprintanswers
\hspace{-.5in} Expression: {\tt 'Ni' * 3}
\part[1] Calculated value: \underline{ 'NiNiNi'  }\\
\part[1] Type: \underline{ {\tt str}  }\\\\
\else
\hspace{-.5in} Expression: {\tt 'Ni' * 3}
\part[1] Calculated value: \makebox[1in]{\hrulefill}\\
\part[1] Type: \makebox[1in]{\hrulefill}\\\\
\fi

\ifprintanswers
\hspace{-.5in} Expression: {\tt 1 // 2}
\part[1] Calculated value: \underline{ 0  }\\
\part[1] Type: \underline{ {\tt int}  }\\\\
\else
\hspace{-.5in} Expression: {\tt 1 // 2}
\part[1] Calculated value: \makebox[1in]{\hrulefill}\\
\part[1] Type: \makebox[1in]{\hrulefill}\\\\
\fi

\ifprintanswers
\hspace{-.5in} Expression: {\tt True and (1 == 2)}
\part[1] Calculated value: \underline{ {\tt False}  }\\
\part[1] Type: \underline{ {\tt bool}  }\\
\else
\hspace{-.5in} Expression: {\tt True and (1 == 2)}
\part[1] Calculated value: \makebox[1in]{\hrulefill}\\
\part[1] Type: \makebox[1in]{\hrulefill}\\
\fi

\end{parts}


\newpage


\question  {\bf Multiple Choice} Circle the letter of the correct choice.

\begin{parts}

\part[2] Given the following code:

\begin{lstlisting}[language=Python]
capitals = {}
capitals['Murica'] = 'Warshington'
capitals['Germany'] = 'Bonn'
capitals['France'] = 'Paris'
capitals['Engalnd'] = 'London'
capitals['Germany'] = 'Berlin'
\end{lstlisting}

What is {\tt capitals['Germany']}?

\begin{choices}
\correctchoice {\tt 'Berlin'}
\choice {\tt 'Sweden'}
\choice {\tt 'Paris'}
\choice {\tt 'London'}
\end{choices}

\part[2] What is {\tt len(set(['A', 'b', 'b', 'a']))}

\begin{choices}
\choice 2
\correctchoice 3
\choice 4
\choice 0
\end{choices}

\part[2] What is wrong with this code:

\begin{lstlisting}[language=Python]
n = 5
while n > 0:
    print(n)
n -= 1
\end{lstlisting}

\begin{choices}
\choice The variable {\tt n} is declared outside the scope of the {\tt while} loop.
\correctchoice The {\tt while} loop never finishes.
\choice The variable {\tt n} is the wrong type.
\choice There is nothing wrong with this code.
\end{choices}

\part[2] What's the value of the expression {\tt ''.join('h a n d s'.split())}

\begin{choices}
\correctchoice {\tt 'hands'}
\choice {\tt 'h a n d s'}
\choice {\tt ['h', 'a', 'n', 'd', 's']}
\choice {\tt None}
\end{choices}


\end{parts}

\newpage

\question {\bf Tracing}

\begin{parts}

Consider the following program:
\begin{lstlisting}[language=Python]
counter = 0;

def incrementCounter():
    global counter
    counter += 1
    return True

if __name__ == '__main__':
    a = True
    b = False;
    if b or incrementCounter():
        print("Boo")
    if (a or b) and incrementCounter():
        print("ya!")
    print(counter)
\end{lstlisting}

\part[5] What is printed when this program is run from the command line?

\begin{solution}[2in]
Boo\\
ya!\\
2
\end{solution}

Consider the following program:
\begin{lstlisting}[language=Python]
mystery = "mnerigpaba"
solved = ""
for i in range(len(mystery) // 2):
    j = -i - 1
    solved += mystery[i] + mystery[j]
print(solved)
\end{lstlisting}

\part[5] What is printed when this program is run from the command line?

\begin{solution}[2in]
manbearpig
\end{solution}


\end{parts}

\newpage

\question {\bf Short Answer}
\begin{parts}

\part[2] What is the value of "abcdefg"[::-1]

\begin{solution}[1in]
\begin{verbatim}
'gfedcba'
\end{verbatim}
\end{solution}

\part[2] Write a list comprehension that returns a list of the first 5 squares where the first square is 1.

\begin{solution}[1in]
\begin{verbatim}
[x * x for x in range(1, 6)]
\end{verbatim}
\end{solution}

\part[2] Write an expression that computes the average of a list of numbers {\tt nums}.

\begin{solution}[1in]
\begin{verbatim}
sum(nums) / len(nums)
\end{verbatim}
\end{solution}

\part[2] Make the dictionary variable, {\tt e2f}, that contains mappings from English words to their French equivalents.  Use these words: dog is chien, cat is chat, and walrus is morse.

\begin{solution}[1in]
\begin{verbatim}
e2f = {'dog': 'chien', 'cat': 'chat', 'walrus': 'morse'}
\end{verbatim}
\end{solution}

\part[2] Write a dictionary comprehension that converts {\tt e2f} to a dictionary from French words to their english equivalents and assigns this new dictionary to a variable {\tt f2e}

\begin{solution}[1in]
\begin{verbatim}
f2e = {f: e for e, f in e2f.items()}
\end{verbatim}
\end{solution}


\end{parts}

\newpage

\question {\bf Complete the Method}

\begin{parts}

\part[5] Fill in the code for the following method that takes a list of numbers and returns the number of even numbers in list argument.  Your code should use a {\tt for} statement.

\begin{verbatim}
def evens(nums):
\end{verbatim}
\begin{solution}[3in]
\begin{verbatim}
    count = 0
    for num in nums:
        if num % 2 == 0:
            count += 1
    return count
\end{verbatim}
\end{solution}

\part[5] Fill in the code for the following method that takes a list of numbers and a number and returns {\tt True} if the list contains the number, {\tt False} otherwise.  You will need a loop, and your loop must not execute more iterations than necessary, and you cannot use {\tt break} or {\tt continue} or the {\tt in} operator.

\begin{verbatim}
def contains(nums, n):
    // Your code goes here
\end{verbatim}
\begin{solution}[3in]
\begin{verbatim}
    found = False
    i = 0
    while i < len(nums) and not found:
        if nums[i] == n:
            found = True
        i += 1
    return found
\end{verbatim}
\end{solution}


\end{parts}


\newpage

\question {\bf Write the method.}  Assume valid input.

\begin{parts}

\part[10] Given a $m \times n$ matrix $\mathbf{A}$:

   \[
   \mathbf{A} = \left[\begin{array}{cccc}
                  A_{11} & A_{12} & \cdots & A_{1n} \\
                  A_{21} & A_{22} & \cdots & A_{2n} \\
                  \vdots & \vdots  & \ddots & \vdots \\
                  A_{m1} & A_{m2} & \cdots & A_{mn} \\
                  \end{array}\right]
   \]

The transpose $\mathbf{A}^T$ is defined as: $\left[\mathbf{A}^T\right]_{ji} = \left[\mathbf{A}\right]_{ij}$.  Think ``the rows of a matrix are the columns of its transpose.''  One way to represent matrices in Python is as a list of lists, for example:
\begin{verbatim}
m = [
    [1, 2, 3],
    [4, 5, 6]
]
\end{verbatim}
Write a method {\tt transpose} that takes a single parameter {\tt m} representing a 2-dimensional matrix as a list of lists and returns its transpose as a list of lists.  Hint: it's possible to do this in one line, but you may use {\tt for} statements instead.

\ifprintanswers
\begin{lstlisting}[language=Java]
def transpose(m):
    mt = [[0] * len(m) for column in m[0]]
    for row in range(len(m)):
        for column in range(len(m[row])):
            mt[column][row] = m[row][column]
    return mt

def transpose2(m):
    return [[row[i] for row in m] for i in range(len(m[0]))]

def transpose3(m):
    return [list(row) for row in zip(*m)]
\end{lstlisting}
\else
\vspace{3in}
\fi


\end{parts}

\newpage

\begin{parts}

\part[5] Write a class {\tt Person} with three instance variables: {\tt name}, {\tt age}, and {\tt email} and two methods:
\begin{itemize}
\item {\tt is\_senior()}, which returns {\tt True} if the {\tt Person} instance's {\tt age} is greater than 59, and
\item {\tt user\_name()}, which returns the user name portion of the instance's {\tt email}, that is, the part before the {\tt @} symbol.
\end{itemize}

\ifprintanswers
\begin{lstlisting}[language=Python]
class Person():
    def __init__(self, name, age, email):
        self.name = name
        self.age = age
        self.email = email

    def is_senior(self):
        return self.age > 59

    def user_name(self):
        return self.email.split('@')[0]
\end{lstlisting}
\else
\vspace{4in}
\fi

\part[5] Write function, {\tt oldest}, that takes a variable number of {\tt Person} (from previous question) parameters (that is, a variable number of single {\tt Person} objects) and returns the oldest {\tt Person} among the arguments.  Assume {\tt oldest} is always called with at least one argument.

\ifprintanswers
\begin{lstlisting}[language=Python]
def oldest(*persons):
    return sorted(persons, key=lambda p: p.age)[-1]
\end{lstlisting}
\else
\vspace{2in}
\fi

\end{parts}

\end{questions}

\end{document}
